\section{Hooks}
\begin{frame}
\frametitle{GIT: Hooks}

Hooks são scripts executados em determinados momentos por comandos do GIT. Encontram-se na pasta \textbf{.git/hooks/}

Para habilitar um hook, nomeie o arquivo com o nome do evento e o proporcione permissões de execução.

\textbf{chmod a+x <file>}

EX.

\textbf{git push}

hooks: pre-commit, post-update
\end{frame}

\begin{frame}
\frametitle{GIT: Exemplo de Hook}

\textbf{.git/hooks/pre-commit}
\scriptsize
\verbatiminput{hook_pre-commit.sh}
\normalsize
\end{frame}

