\section{Trabalhando com o GIT}
\begin{frame}
\frametitle{Workflow}

\pgfdeclareimage[width=5cm,height=6cm]{Git_data_flow_simplified}{Git_data_flow_simplified}
\pgfuseimage{Git_data_flow_simplified}<1>

\end{frame}


\begin{frame}
\frametitle{Comandos de uso diário}


\begin{itemize}
\item git add
\item git commit
\item git diff
\item git log
\item git push
\item git pull
\item git fetch
\item git rebase
\item git show
\item gitk
\end{itemize}


\end{frame}

\begin{frame}
\frametitle{GIT: Comandos de transporte de dados}

\pgfdeclareimage[width=6cm,height=7cm]{git-transport.png}{git-transport.png}
\pgfuseimage{git-transport.png}<1>

\end{frame}

\begin{frame}
\frametitle{Workflow: Publicando alterações sem repositório}

\pgfdeclareimage[width=5cm,height=6cm]{git-workflow-rsync.png}{git-workflow-rsync.png}
\pgfuseimage{git-workflow-rsync.png}<1>

\end{frame}

\begin{frame}
\frametitle{git diff}


Apenas repara em arquivos "rastreados".

\begin{itemize}
\item \textbf{git diff}: diferenças entre a área de trabalho (workspace) e o índice (index)
\item \textbf{git diff --cached}: diferenças entre índice (index) e o repositório local
\item \textbf{git diff HEAD}: diferenças entre a área de trabalho (workspace) e o repositório local.
\end{itemize}
\end{frame}

\begin{frame}
\frametitle{História}

\begin{itemize}
\item \textbf{git log}
\item \textbf{git log <commit ID>}
\begin{itemize}
\item shows history before that commit
\item ex. git log HEAD
\item ex.  git log f74e955797c273c0398494ab33bf6e965c3b97a4
\end{itemize}
\item \textbf{git log -p}
\begin{itemize}
\item shows log as a series of patches
\end{itemize}
\end{itemize}

\end{frame}


\begin{frame}
\frametitle{GIT: trabalando com alterações}


\begin{itemize}
\item \textbf{git add <file>}: adds a file to the Index
\item \textbf{git commit}:  commits added changes to the local repo
\end{itemize}

Detalhes importantes:

\begin{itemize}
\item Arquivos que não foram adicionados ao índice (index) não são incluídos no commit
\item Adicionando e fazendo commits para cada arquivo, individualmente, torna o gerenciamento do código mais fácil pois reduz o tamanho de cada commit/patch.
\end{itemize}

\end{frame}
