\section{GIT}

\begin{frame}
\frametitle{O que é GIT?}

\begin{block}{Linus Torvalds}
\textbf{Sou um bastardo egoísta e nomeio todos os meus projetos em homenagem a mim mesmo. Primeiro Linux, agora GIT.}
\end{block}

\begin{itemize}
\item \textbf{Linux}: Híbrido de Linus e Mimix
\item \textbf{Git}: gíria britânica que descreve uma pessoa idiota ou sem valor.
\end{itemize}

\end{frame}

\begin{frame}
\frametitle{O que é GIT?}
GIT é um Sistema de Controle de Versões Disitribuído (Distributed Revision Control - DRC) e Gerenciamento de Código Fonte(Source Code Manager SCM).
\begin{itemize}
\item Ênfase em velocidade
\item Cada diretório de trabalho é um repositório completo - Não depende de acesso à rede ou um servidor central.
\item GPL2
\end{itemize}
\end{frame}


\begin{frame}
\frametitle{Objetivos}

\begin{itemize}
\item Facilitar desenvolvimento distribuído
\item Escalável - suportar milhares de usuários
\item Execução rápida e eficiente
\item Manter integridade e confiança
\item Obrigar responsabilidades (enforce accountability)
\item Imutabilidade – depois de criados os objetos não podem ser alterados.
\item Transações atômicas – alterações diferentes mas relacionadas são executadas como um todo ou não são executadas.
\item Suportar e incentivar desenvolvimento ramificado (Branched Development)
\item Repositórios completos
\item Design interno limpo
\item Livre, “como em liberdade”
\end{itemize}

\end{frame}

\begin{frame}
\frametitle{Mas quem usa GIT?}

\pgfdeclareimage[width=12cm,height=6cm]{who_uses_git}{who_uses_git}
\pgfuseimage{who_uses_git}<1>

\end{frame}

\begin{frame}
\frametitle{Mas quem hospeda repositórios GIT?}

\pgfdeclareimage[width=12cm,height=6cm]{hosting}{hosting}
\pgfuseimage{hosting}<1>

\end{frame}

\begin{frame}
\frametitle{A Filosofia do GIT}
\begin{itemize}
\item Commit early, commit often (Release early, release often - Lei de Linus);
\item Cada commit representa uma idéia e uma alteração;
\begin{itemize}
\item Facilita a leitura das alterações (patches)
\item Facilita reverter alterações indesejadas no futuro
\end{itemize}
\item O diretório de trabalho, index e repositório são rascunhos, uma laboratório, e não o repositório oficial do projeto.
\end{itemize}
\end{frame}

