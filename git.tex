\documentclass{beamer}
\usepackage[utf8x]{inputenc}
\usepackage{ucs}
\usepackage[portuguese]{babel}
\usepackage[T1]{fontenc}
\usepackage{amsmath}
\usepackage{amsfonts}
\usepackage{amssymb}
\usepackage{makeidx}
\usepackage{graphicx}
\usepackage{lmodern}
\usepackage{verbatim}

\usetheme{CambridgeUS}

%\AtBeginSection[]
%{
%\begin{frame}
%\frametitle{Sumário}
%\tableofcontents[currentsection]
%\end{frame}
%}

\pgfdeclareimage[height=.5cm]{logo}{logo}
\logo{\pgfuseimage{logo}}


\author{Fernando Michelotti}
\title{Programação Não Linear com GIT}
\begin{document}

\frame{\titlepage}
\section{Sumario}

\frame{\tableofcontents}

\section{GIT: SCM ou 'the information manager from hell'}

\begin{frame}
\frametitle{O que é GIT?}

\begin{block}{Linus Torvalds}
\textbf{Sou um bastardo egoísta e nomeio todos os meus projetos em homenagem a mim mesmo. Primeiro Linux, agora GIT.}
\end{block}

\begin{itemize}
\item \textbf{Linux}: Híbrido de Linus e Mimix
\item \textbf{Git}: gíria britânica que descreve uma pessoa idiota ou sem valor.
\end{itemize}

\end{frame}

\begin{frame}
\frametitle{O que é GIT?}

GIT é um Sistema de Controle de Versões Disitribuído (Distributed Revision Control - DRC) e Gerenciamento de Código Fonte(Source Code Manager SCM).
\begin{itemize}
\item Ênfase em velocidade
\item Cada diretório de trabalho é um repositório completo - Não depende de acesso à rede ou um servidor central.
\item GPL2
\end{itemize}
\end{frame}

\section{SCMs}

\begin{frame}
\frametitle{O que é um SCM?}

História de SCMs:

\textbf{SCCS – Source Code Control System}

Criado no Início dos anos 1970 por M.J. Rochkind

Início de conceitos ainda pertinentes nos SCMs atuais. ex.: repositório

Problema: arquivo de LOCK - enquanto um usuário trabalha outro não tem acesso ao arquivo

\textbf{RCS – Revision Control System}

Criado no início dos anos 1980 por Walter Tichy

\end{frame}
\begin{frame}
\frametitle{SCM: Histórico}

\textbf{CVS – Concurrent Version System}

Criado em 1986 por Dick Grune.

Extendeu e modificou o modelo do RCS

Altera sistema de lock de forma a permitir que várias pessoas possam trabalhar no mesmo arquivo concorrentemente. Introduzido o conceito de mescla (merge).

\textbf{SVN - Subversion }

Criado com o intuito de corrigir as falhas de arquitetura do CVS

Introduziu:
\begin{itemize}
\item Commit Atomico
\item Melhor suporte a galhos (branches)
\end{itemize}


\end{frame}

\begin{frame}
\frametitle{SCM: Histórico}

\textbf{BitKeeper e Mercurial}

Distanciam-se do modelo dos sistemas anteriores ao eliminar o repositório central. 

Armazenamento distribuído:

\begin{itemize}
\item cada desenvolvedor possui sua própria cópia completa e compartilhável
\item derivado do modelo P2P (peer-to-peer)
\end{itemize}

\textbf{Mercurial e Monotone}

Utilizam hash para identificar o conteúdo dos arquivos 

Influenciou a arquitetura do GIT.
\end{frame}

\section{GIT: a origem}

\begin{frame}
\frametitle{Kernel do Linux}

\begin{itemize}
\item Utilizou tarbals e arquivos de patch.
\item Controvérsia com a utilização do BitKeeper (proprietário)
\item Problema de Licença do BitKeeper - Larry McVoy removeu a licença de utilização do BitKeeper para projetos de código aberto depois de acusar Andrew Tidwell de tentar executar engenharia reversa no BitKeeper.
\item Linus Torvalds faz acordo com McVoy e abandona BitKeeper no desenvolvimento da kernel.
\item Linus analisa SCMs abertos e conclui que nenhum deles respondem à suas necessidades. Entre as opções de voltar a tarballs / patches e criar um SCM próprio opta pela segunda.
\end{itemize}
\end{frame}

\begin{frame}
\frametitle{Linux e CVS}

\begin{block}{Linus Torvalds sobre CVS}
Veja o CVS como um exemplo o que não fazer; quando em dúvida, tome a decisão exatamente oposta..
Pelos primeiros 10 anos de desenvolvimento da kernel, nós literalmente utilizamos tarballs e patches, o qual é um sistema de gerenciamento de código muito superior do que CVS, mas eu acabei utilizando CVS por 7 anos em uma
empresa comercial e eu o odeio com paixão. Quando digo que eu odeio CVS com paixão, eu também devo dizer que se há algum usuário de SVN na audiência, vocês podem desejar sair. Porque meu ódio pelo CVS significa que
vejo o Subversion como o projeto mais sem sentido já iniciado. O slogan do Subversion foi por um tempo “CVS feito certo”, ou algo assim, e se você começar com esse slogan, não há lugar algum que você possa ir. Não existe forma de fazer CVS corretamente.
\end{block}
\end{frame}

\begin{frame}
\frametitle{Linux e BitKeeper}
\begin{block}{Torvalds falando sobre BitKeeper}
BitKeeper não foi apenas o primeiro sistema de gerenciamento de código que eu considerei merecedor de ser utilizado, também foi o sistema de gerenciamento de código que me ensinou porque utilizá-los, como você pode realmente fazer as coisas. Então git em de muitas formas, mesmo que tecnicamente seja muito diferente de BitKeeper (o que era outra meta do projeto, pois eu desejava deixar bem claro que não era um clone do BitKeeper), muitos dos fluxos utilizados no git vem diretamente dos fluxos nós aprendemos do BitKeeper.
\end{block}
\end{frame}

\begin{frame}
\frametitle{Objetivos}

\begin{itemize}
\item Facilitar desenvolvimento distribuído
\item Escalável - suportar milhares de usuários
\item Execução rápida e eficiente
\item Manter integridade e confiança
\item Obrigar responsabilidades (enforce accountability)
\item Imutabilidade – depois de criados os objetos não podem ser alterados.
\item Transações atômicas – alterações diferentes mas relacionadas são executadas como um todo ou não são executadas.
\item Suportar e incentivar desenvolvimento ramificado (Branched Development)
\item Repositórios completos
\item Design interno limpo
\item Livre, “como em liberdade”
\end{itemize}

\end{frame}

\section{GIT}
\begin{frame}
\frametitle{Mas quem usa GIT?}

\pgfdeclareimage[width=12cm,height=6cm]{who_uses_git}{who_uses_git}
\pgfuseimage{who_uses_git}<1>

\end{frame}

\begin{frame}
\frametitle{Mas quem hospeda repositórios GIT?}

\pgfdeclareimage[width=12cm,height=6cm]{hosting}{hosting}
\pgfuseimage{hosting}<1>

\end{frame}

\begin{frame}
\frametitle{A Filosofia do GIT}
\begin{itemize}
\item Commit early, commit often (Release early, release often - Lei de Linus);
\item Cada commit representa uma idéia e uma alteração;
\begin{itemize}
\item Facilita a leitura das alterações (patches)
\item Facilita reverter alterações indesejadas no futuro
\end{itemize}
\item O diretório de trabalho, index e repositório são rascunhos, uma laboratório, e não o repositório oficial do projeto.
\end{itemize}
\end{frame}

\section{Trabalhando com o GIT}
\begin{frame}
\frametitle{Workflow}

\pgfdeclareimage[width=5cm,height=6cm]{Git_data_flow_simplified}{Git_data_flow_simplified}
\pgfuseimage{Git_data_flow_simplified}<1>

\end{frame}


\begin{frame}
\frametitle{Comandos de uso diário}


\begin{itemize}
\item git add
\item git commit
\item git diff
\item git log
\item git push
\item git pull
\item git fetch
\item git rebase
\item git show
\item gitk
\end{itemize}


\end{frame}

\begin{frame}
\frametitle{GIT: Comandos de transporte de dados}

\pgfdeclareimage[width=6cm,height=7cm]{git-transport.png}{git-transport.png}
\pgfuseimage{git-transport.png}<1>

\end{frame}

\begin{frame}
\frametitle{Workflow: Publicando alterações sem repositório}

\pgfdeclareimage[width=5cm,height=6cm]{git-workflow-rsync.png}{git-workflow-rsync.png}
\pgfuseimage{git-workflow-rsync.png}<1>

\end{frame}

\begin{frame}
\frametitle{git diff}


Apenas repara em arquivos "rastreados".

\begin{itemize}
\item \textbf{git diff}: diferenças entre a área de trabalho (workspace) e o índice (index)
\item \textbf{git diff --cached}: diferenças entre índice (index) e o repositório local
\item \textbf{git diff HEAD}: diferenças entre a área de trabalho (workspace) e o repositório local.
\end{itemize}
\end{frame}

\begin{frame}
\frametitle{História}

\begin{itemize}
\item \textbf{git log}
\item \textbf{git log <commit ID>}
\begin{itemize}
\item shows history before that commit
\item ex. git log HEAD
\item ex.  git log f74e955797c273c0398494ab33bf6e965c3b97a4
\end{itemize}
\item \textbf{git log -p}
\begin{itemize}
\item shows log as a series of patches
\end{itemize}
\end{itemize}

\end{frame}


\begin{frame}
\frametitle{GIT: trabalando com alterações}


\begin{itemize}
\item \textbf{git add <file>}: adds a file to the Index
\item \textbf{git commit}:  commits added changes to the local repo
\end{itemize}

Detalhes importantes:

\begin{itemize}
\item Arquivos que não foram adicionados ao índice (index) não são incluídos no commit
\item Adicionando e fazendo commits para cada arquivo, individualmente, torna o gerenciamento do código mais fácil pois reduz o tamanho de cada commit/patch.
\end{itemize}

\end{frame}

\begin{frame}
\frametitle{GIT: Alterando a história}

De acordo com a Lei de Murphy: Depois de um commit você irá encontrar bugs em seu código.

Você pode fazer um novo commit com a correção do bug

OU 

Você pode "amend" (aperfeiçoar) o commit anterior

Corrija seu código e execute:

\textbf{git add <file>}

para adicionar seu código ao index e

\textbf{git commit --amend}

para modificar o commit no repositório local.
\end{frame}


\begin{frame}
\frametitle{GIT: Alterando a história}

Uma reescrita completa do histórico:

\textbf{git rebase -i <commit ID>}

\begin{itemize}
\item Pode reordenar commits
\item Pode editar commits
\item Pode "espremer" (squash) um commit em outro
\end{itemize}


\end{frame}

\section{Hooks}
\begin{frame}
\frametitle{GIT: Hooks}

Hooks são scripts executados em determinados momentos por comandos do GIT. Encontram-se na pasta \textbf{.git/hooks/}

Para habilitar um hook, nomeie o arquivo com o nome do evento e o proporcione permissões de execução.

\textbf{chmod a+x <file>}

EX.

\textbf{git push}

hooks: pre-commit, post-update
\end{frame}

\begin{frame}
\frametitle{GIT: Exemplo de Hook}

\textbf{.git/hooks/pre-commit}
\verbatiminput{hook_pre-commit.sh}
\end{frame}

\section{GIT Blame}
\begin{frame}
\frametitle{GIT: encontrando o culpado}


\begin{itemize}
\item \textbf{git blame <file>}: mostra quem fez o commit de cada linha do arquivo
\item \textbf{git blame <file> <commit ID>}: mostra o histórico das linhas antes do commit
\end{itemize}
\end{frame}

\section{GIT / SVN}

\begin{frame}
\frametitle{GIT/SVN: Commit}
\begin{tabular}{ l r }
git init & svnadmin create repo \\
git add . & svn import file://repo \\
git commit & . \\
git diff & svn diff | less \\
git diff rev path & svn diff -rrev path \\
git apply & patch -p0 \\
git status & svn status \\
git checkout path & svn revert path \\
git add file & svn add file \\
git rm file  & svn rm file \\
git mv file & svn mv file \\
git commit -a & svn commit \\
\end{tabular}
\end{frame}

\begin{frame}
\frametitle{GIT/SVN: Navegando}

\begin{tabular}{ l r }
git log & svn log | less \\
git blame file & svn blame file \\
git show rev:path/to/file & svn cat url \\
git show rev:path/to/directory & svn list url
git show rev & svn log -rrev url ; svn diff -crev url \\
\end{tabular}

\end{frame}

\begin{frame}
\frametitle{GIT/SVN: Tagging e Branching}
\begin{tabular}{ l r }
git tag -a name & svn copy http://example.com/svn/trunk http://example.com/svn/tags/name \\
git tag -l & svn list http://example.com/svn/tags/ \\
git show tag & svn log --limit 1 http://example.com/svn/tags/tag \\
git branch branch  & svn copy http://example.com/svn/trunk http://example.com/svn/branches/branch  \\
git checkout branch & svn switch http://example.com/svn/branches/branch \\
git branch & svn list http://example.com/svn/branches/ \\
git checkout rev  & svn update -r rev \\
git checkout prevbranch & svn update \\
\end{tabular}
\end{frame}

\begin{frame}
\frametitle{GIT/SVN: Merging}
\begin{tabular}{ l r }
git merge branch & (assuming the branch was created in revision 20 and you are inside a working copy of trunk) svn merge -r 20:HEAD http://example.com/svn/branches/branch \\
git cherry-pick rev & svn merge -c rev url \\
\end{tabular}
\end{frame}

\begin{frame}
\frametitle{GIT/SVN: Remote}

\begin{tabular}{ l r }
git clone url & svn checkout url \\
git checkout --track -b branch origin/branch & svn switch url \\
git pull & svn update \\
git push remote & \\
git send-email & \\
\end{tabular}

\end{frame}

% http://git.or.cz/course/svn.html

\end{document}
